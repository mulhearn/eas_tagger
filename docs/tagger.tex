\documentclass[12pt]{article}

\usepackage[dvips,letterpaper,margin=0.75in,bottom=0.5in]{geometry}
\usepackage{cite}
\usepackage{slashed}
\usepackage{graphicx}
\usepackage{amsmath}
\usepackage{amssymb}
\usepackage{braket}
\usepackage{mathtools}
\usepackage{tikz}
\usepackage{bm}
\usepackage{hyperref}
\usepackage{microtype}
\usepackage{geometry}

\begin{document}

\title{Plan For EAS Tagger}
\author{Michael Mulhearn}

\maketitle

\section{Preliminaries:  Differential Intensity of Extensive Air Showers}

\begin{figure}[htbp]
    \centering
    \includegraphics[width=0.9\textwidth]{figs/fluxes.pdf}
    \caption{Incident differential intensity, from the CRDB.}
    \label{fig:cridi}
\end{figure}

The differential intensity $J$ of cosmic rays incident at the top of
the atmosphere (TOA) is defined as
$$J = \frac{dN}{dE \, dA \, dt \, d\Omega}$$
where $E$ is the energy of the incident primary and $t$ is time.
The quantities $d\Omega$ and $dA$ are the infinitesimal solid angle,
relative to normal, and the infinitesimal area, both {\bf at the TOA}.
The intensity has units:
$$[J] = \left[ \frac{1}{\rm \epsilon \; m^2 \; sr \; s} \right]$$
where $\epsilon$ is some energy scale, e.g. 1~GeV, and sr is a
dimensionless reminder to integrate over solid angle.  The incident
differential intensity at the top of the atmosphere is independent of
direction and follows a power law:
\begin{equation}
J(E) = \alpha \left( \frac{E}{\epsilon} \right)^{-\gamma}
\end{equation}
with $\gamma ~\sim 3$.  The power law is only approximate.  Our
results below include an estimate for a minimum energy, and we expect
only a narrow energy region above this minimum will contribute
significantly.  The parameters for the power law should therefore be
taken for that energy range.

The differential intensity as tabulated by the Cosmic Ray Database
(CRDB) is shown in Fig.~\ref{fig:cridi}.  When plotting a power law of form:
$$\frac{dy}{dx} = \alpha x^{-\gamma}$$
is often preferable to use a log-log scale, so that the power law becomes the linear relationship:
$$\log \left( \frac{dy}{dx} \right) = -\gamma \log(x) + \log(\alpha)$$
It also useful to set the $y$-axis to:
$$x \frac{dy}{dx} = \frac{dy}{d(\log(x))}$$
which removes any dependence\footnote{Notice that
$d(log(x/X)) = d(log(x))$ for any constant $X$.}  on the units of $x$,
and is appropriate for binning the $x$ parameters on a log scale.
Furthermore, for a decreasing power lower, a calculation shows:
$$\int_{a}^\infty \frac{dy}{dx} dx = \frac{1}{\gamma - 1} \left. x \frac{dy}{dx} \right|_{x=a}$$
so the $y$-axis is proportional to the integrated rate.  Returning to Fig.~\ref{fig:cridi}, note that indeed:
$$10^{-5} \cdot 24 \cdot 60 \cdot 60 = 0.86$$
is indeed nearly 1 muon per day.

Next we are interested in the flux of incident cosmic ray particles
when projected (without any attenuation or scattering, this is just
for accounting) to the surface of the earth.  We first approximate
the earth as an infinite plane (appropriate for our thin atmosphere,
as we shall see) and calculate the angle-integrade differential flux as:
simply:
$$F = \int  J \cos \theta d\Omega = 2\pi J \int_0^{\pi/2} \cos\theta \, \sin\theta \, d\theta$$
where we integrate only over the upper hemisphere, have included
a factor of $\cos\theta$ for a horizontal surface, and used the fact
that $J$ is independent of direction.  Computing the integral we find that:
\begin{equation}
F = \pi J
\end{equation}

For an atmosphere of height $h$ and earths radius $R$ and impact
parameter $b$, at TOA we have incident angle $\alpha$ and:
$$\sin \alpha = \frac{b}{R+h}$$
and at ground we have incident nalge $beta$ and:
$$\sin \beta = \frac{b}{R}$$
so that:
\begin{equation}
(R+h) \sin \alpha = R \sin \beta
\end{equation}
It is left as an exercise to compute the Jacobian:
$$\frac{d\Omega_\alpha}{d\Omega_\beta} = \left( \frac{R}{R+h} \right)^2 \frac{\cos \beta}{\sqrt{1 - \left( \frac{R}{R+h} \right)^2 \sin^2\beta}}$$
and show that the resulting incident flux per unit horizontal area (now accounting for earth's shape) is:
$$F = 0.92 \, \pi \, J $$
for TOA at $100~\rm km$.  Note that $\alpha$ and $\beta$ will be
recycled as parameters after this exercise!

\section{Local Muon Flux from Extensive Air Showers}

A critical consideration for the feasibility of studying extensive air
showers in the lab, using a local network of cell phones, is the rate
at which the muon flux in the lab is detectable by the phone network.
A typical phone has an area times efficiency of
$$A_{\rm eff} = 2 \times 10^{-5}~\rm m^2$$
We define a detectable muon flux $\Phi$ as the density at which a
network of $N_{\rm P}$ phones can be expected to register at least one
hit.  So for $N_{\rm P}=1000$, we have:
\begin{equation}
\Phi_{\rm D} = \frac{1}{A_{\rm eff}~N_{\rm P} } = 50~{\rm muons} / {\rm m^2}
\end{equation}

We saw in the preceding section that the incident flux of cosmic rays follows a power law:
$$ J(E) = \alpha \left( \frac{E}{\epsilon} \right)^{-\gamma} $$
and the angle-integrated differential flux:
$$ F(E) = 0.92 \; \pi \; J(E) $$
where we can set 0.92 to one if we want to approximate the earth as an
infinite plane.  The number of muons $N$ produced by a shower of energy $E$ also follows a power law:
\begin{equation}
N(E) = \beta \left( \frac{E}{\epsilon} \right)^{\eta}
\end{equation}
Suppose that these $N$ muons have lateral profile give by:
$$\Phi(r) = \Phi(0) \exp(-r / \lambda)$$
where $\lambda$ is the length scale of the muon lateral profile.  The condition:
$$\int_0^\infty \; \Phi(r) \, 2 \pi r \, dr = N$$
implies that:
\begin{equation}
\Phi(0) = \frac{N(E)}{2 \pi \lambda^2}
\end{equation}
and
\begin{equation}
\Phi(r) =  \frac{N(E)}{2 \pi \lambda^2} \; \exp(-r / \lambda)
\end{equation}
The radius at which the flux $\Phi$ drops to $\Phi_{\rm D}$ is:
$$r = -\lambda \ln \left( \frac{2 \pi \lambda^2 \Phi_{\rm D}}{N} \right)$$
which means the area $A_{\rm D}$ over which there is a detectable muon flux is give by:
$$A_{\rm D}(E) = \pi \lambda^2 \; \left( \ln \frac{2 \pi \lambda^2 \Phi_{\rm D}}{N(E)} \right)^2$$
which is only valid ($A_{rm D} > 0$) when:
$$\Phi(0) = \frac{N(E)}{2\pi\lambda^2} > \Phi_D$$
which sets a minimum energy, as we shall see.  Using the power law, note that:
$$\frac{N(E)}{\Phi_{\rm D} \, 2\pi\lambda^2}
= \frac{\beta}{\Phi_{\rm D} \, 2\pi\lambda^2} \left( \frac{E}{\epsilon}\right)^{\eta} = \left( \frac{E}{\kappa \epsilon}\right)^{\eta}$$
where
\begin{equation}
\kappa = \left( \frac{\Phi_{\rm D} 2 \pi \lambda^2}{\beta}\right)^{1/\eta}
\end{equation}
and so:
\begin{equation}
A_{\rm D}(E) = \pi \lambda^2 \eta^2 \; \left( \ln \frac{E}{\kappa \epsilon} \right)^2
\end{equation}
which is only valid for $E > \kappa \epsilon$ (which you can verify is equivalent to requiring that $\Phi(0) > \Phi_{\rm D}$).

We are interested in the quanity:
\begin{equation}
\frac{dN}{dE dt} = F(E) A_{\rm D}(E) = (0.92) \pi J(E) A_{\rm D}(E)  
\end{equation}
Plugging in the results below we obtain our main result:
\begin{equation}
\frac{dN}{dE \, dt} = (0.92) \alpha \, (\pi \lambda \eta)^2 \, \left( \frac{E}{\epsilon}\right)^{\eta-\gamma}
\left( \ln \frac{E}{\kappa \epsilon}\right)^2
\end{equation}
which we need only integrate to calculate the rate at which a point on the surface of the earth has a local muon flux exceeding $\Phi_D$.  Defining:
\begin{equation}
f(\kappa) = \int_\kappa^\infty x^{\eta-\gamma} \left( \ln \frac{x}{\kappa}\right)^2 dx
\end{equation}
we can write the rate as:
\begin{equation}
\frac{dN}{dt} = (0.92) \alpha \epsilon \, (\pi \lambda \eta)^2 f(\kappa)  
\end{equation}




\section{Introduction}



\begin{figure}[htbp]
    \centering
    \includegraphics[width=0.9\textwidth]{figs/eas.png}
    \caption{Extensive air showers, from Chapter 16 of ``Gaissner Cosmic Rays and Particle Physics''}
    \label{fig:eas}
\end{figure}

\begin{itemize}
 \item Using rates from CRDB.  
 \item Using Whitesons formula for the number of muons:
$$N_{\mu} = 0.028 (E / 1~{\rm GeV})^{0.93}$$ Note that for $E=10^{19}~\rm
   eV$, this formular predicts $5\times10^7$ in good agreement with
   LHS of Fig.~\ref{fig:eas}
 \item Muons have typical lateral extent of $\lambda$ greater than Moliere radius.  Assuming:
$$\rho(r) = \frac{N}{2\pi\lambda^2} \ \exp(-x/\lambda)$$
then $\lambda=220$ puts $\rho(1 {\rm km}) = 2 / \rm m^2$ as in figure. 
  
 \item Aim for $E \ge 10^{18}~\rm eV$.

 \item Expected rate is:
$$ R = \frac{1}{\rm km^2 \, day \, sr} \, \pi (\pi \lambda^2) = 0.44 \, / \, \rm day$$
 \item Total muons:
   $$N_\mu = 6.6 \times 10^6$$
 \item And density at center:
   $$\rho(0) = \frac{N_\mu}{2 \pi \lambda^2} = 21 /~\rm m^2$$
 \item $A\epsilon = 2\times 10^{-5}$, for 1000 phones:  $0.02~\rm m^2$
 \item Muons hits in phone per shower:  $0.42$ per shower.
 \item Expect a phone hit coincident with tagger more than once per week.
 \item Taggers of size $0.1 m^2$ should be well suited.
\end{itemize}

\end{document}
