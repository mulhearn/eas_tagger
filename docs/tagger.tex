\documentclass[12pt]{article}

\usepackage[dvips,letterpaper,margin=0.75in,bottom=0.5in]{geometry}
\usepackage{cite}
\usepackage{slashed}
\usepackage{graphicx}
\usepackage{amsmath}
\usepackage{amssymb}
\usepackage{braket}
\usepackage{mathtools}
\usepackage{tikz}
\usepackage{bm}
\usepackage{hyperref}
\usepackage{microtype}
\usepackage{geometry}

\begin{document}

\title{Plan For EAS Tagger}
\author{Michael Mulhearn}

\maketitle

\section{Preliminaries:  Differential Intensity of Extensive Air Showers}

\begin{figure}[htbp]
    \centering
    \includegraphics[width=0.9\textwidth]{figs/fluxes.pdf}
    \caption{Incident differential intensity ($dJ/dE$), as presented
      by the Cosmic Ray Database (CRDB).}
    \label{fig:cridi}
\end{figure}

The differential\footnote{Take care that some sources use J for this
quanity, but we are following the more explicit notation of e.g. the
CRDB} intensity $dJ/dE$ of cosmic rays incident at the top of the
atmosphere (TOA) is defined as
$$dJ/dE = \frac{dN}{dE \, dA \, dt \, d\Omega}$$
where $E$ is the energy of the incident primary and $t$ is time.
The quantities $d\Omega$ and $dA$ are the infinitesimal solid angle,
relative to normal, and the infinitesimal area, both {\bf at the TOA}.
The differential intensity has units:
$$[dJ/dE] = \left[ \frac{1}{\rm energy \; m^2 \; sr \; s} \right]$$
where sr is a dimensionless reminder to integrate over solid angle.  The incident
differential intensity at the top of the atmosphere is independent of
direction and follows a power law:
\begin{equation}
\frac{dJ}{dE} = \frac{\alpha}{\epsilon} \left( \frac{E}{\epsilon} \right)^{-\gamma}
\end{equation}
where $\epsilon$ is a reference energy and $\gamma ~\sim 3$.  Note that
$$\alpha = \epsilon \left. \frac{dJ}{dE}\right|_\epsilon$$
depends on the reference energy choosen but does not depend on the
energy units choosen, which will be convenient in what follows
The energy-integrated intensity above some energy $E$ is:
\begin{equation}
\label{eqn:je}
J(E) \; \equiv \;
\int_{E}^{\infty} \frac{dJ}{dE} \, dE \; = \;
\frac{\alpha}{\gamma-1} \left( \frac{E}{\epsilon} \right)^{(1-\gamma)}
\end{equation}

Power laws quickly take us across many order of magnitude, so it is helpful
to define and work with the dimensionless log-energy variable:
\begin{equation}
  \label{eqn:defu}
  u \equiv \ln \left( \frac{E}{\epsilon}\right)
  \implies E = \epsilon \, e^u
\end{equation}
from which it follows that:
\begin{equation}
du = \frac{dE}{E}
\end{equation}
and
$$\frac{dJ}{du} = E \frac{dJ}{dE}$$
We can therefore write the power law as:
\begin{equation}
\frac{dJ}{du} = \alpha \exp((1-\gamma)\, u)
\end{equation}
direct integration or substituting into Equation~\ref{eqn:defu} into Equation \ref{eqn:je}
yields:
\begin{equation}
J(u) = \frac{\alpha}{\gamma-1} \exp((1-\gamma)\, u) = \frac{1}{1-\gamma} \frac{dJ}{du}
\end{equation}
Note also that:
$$
\ln \frac{dJ}{du} = \ln(\alpha) + (1-\gamma) u
$$
which is a linear relationship with slope $1-\gamma$.

These results show why differential intensity is often presented as $E
dJ/dE = dJ/du$ vs $E$ using a log-log scale, such as in
Fig.~\ref{fig:cridi}.  The power law is a linear relationship on a
log-log scale, with a compressed scale.  Notice that the $y$-axis
drops two orders of magnitude for every one order of magnitude
increase of the $x$-axis, from which we read off:
$$ 1 - \gamma = -2 \implies \gamma = 3$$
Furthermore:
$$J(E) = \frac{1}{1-\gamma} E \frac{dJ}{dE}$$
so the $y$ axis is proportional to the energy-integrated intensity
above $E$.  Also, the grey bands are where you would expect,
e.g. there are $10^5$ seconds in a day, so the once-per-day band
starts at $10^-5$, although the factor of $\gamma-1$ has been
neglected.  Lastly, our definition of $\alpha$ allow us to read it
directly from the plot: it is the $y$-value obtained by the curve at
the reference energy $\epsilon$.

\section{Shower Energy generation}

The power law is only approximate.  Our results below include an
estimate for a minimum shower energy, and we expect that only a narrow
energy region above this minimum will contribute significantly to our
experimental results.  The parameters for the power law should
therefore be taken as the energy range appropriate for our experiment.

We will want to simulate showers by drawing a random variable $u$, the
dimensionless log-energy of the shower, following the distribution
$dJ/du$.  This is equivalent to drawing a primary energy $E$ according
to the power-law distribution $dJ/dE$, but is less susceptible to
round off errors.  To avoid the divergence at $E=0$, we set a minimum shower energy $E_a$ and define:
$$a \equiv \ln (E_a / \epsilon)$$
The cumulative distribution function is:
$$P(u) = \frac{J(a) - J(u)}{J(a)}$$
so
\begin{equation}
P(u) = 1 - \exp\left( (1-\gamma)\,(u-a) \right)
\end{equation}
which is easily invertible.  Therefore, to generate a random variable $u$, throw $0 \leq x < 1$
and calculate:
\begin{equation}
u = a - \frac{1}{\gamma-1} \ln (1-x)
\end{equation}
After generating $N$ such shows, each should have a weight:
$$w = \frac{1}{N} \, J(a) = \frac{1}{N} \, \frac{\alpha}{1-\gamma} \, \exp\left( (1-\gamma) \, a \, \right)$$


\section{Preliminaries: Flat-Surface Flux}

We are interested in the flux of incident cosmic ray particles
when projected (without any attenuation or scattering, this is just
for accounting) to the surface of the earth.  We first approximate
the earth as an infinite plane (appropriate for our thin atmosphere,
as we shall see) and calculate the flat-surface differential flux as:
simply:
$$F = \int  J \cos \theta d\Omega = 2\pi J \int_0^{\pi/2} \cos\theta \, \sin\theta \, d\theta$$
where we integrate only over the upper hemisphere, have included
a factor of $\cos\theta$ for a flat horizontal surface, and used the fact
that $J$ is independent of direction.  Computing the integral we find that:
\begin{equation}
F = \pi J
\end{equation}
or equivently:
\begin{equation}
\frac{dF}{dE} = \pi \frac{dJ}{dE}
\end{equation}

Our earth has atmosphere of height $h$ and radius $R$.  An incident
cosmic ray primary has some impact parameter $b$, which is the same whethere measured at TOA or on the ground.  At TOA we have incident angle $\theta_h$ and:
$$\sin \theta_h = \frac{b}{R+h}$$
while at ground we have incident angle $\theta_g$ and:
$$\sin \theta_g = \frac{b}{R}$$
so that:
\begin{equation}
(R+h) \sin \theta_h = R \sin \theta_g
\end{equation}
The Jacobian:
$$\frac{d\Omega_h}{d\Omega_g} = \left( \frac{R}{R+h} \right)^2 \frac{\cos \theta_g}{\sqrt{1 - \left( \frac{R}{R+h} \right)^2 \sin^2\theta_g}}$$
and the resulting incident flat-surface differential flux (now accounting for earth's shape) is:
$$F = 0.92 \, \pi \, J $$
for TOA at $100~\rm km$.

\section{Local Muon Flux from Extensive Air Showers}

A critical consideration for the feasibility of studying extensive air
showers in the lab, using a local network of cell phones, is the rate
at which the muon flux in the lab is detectable by the phone network.
A typical phone has an area times efficiency of
$$A_{\rm eff} = 2 \times 10^{-5}~\rm m^2$$
We define a detectable muon flux $\Phi$ as the density at which a
network of $N_{\rm P}$ phones is expected to register at least one hit
on average.  So for $N_{\rm P}=1000$, we have:
\begin{equation}
\Phi_{\rm D} = \frac{1}{A_{\rm eff}~N_{\rm P} } = 50~{\rm muons} / {\rm m^2}
\end{equation}

We saw in the preceding section that the incident flux of cosmic rays follows a power law, and the differential intensity is given by 
$$ \frac{dJ}{dE} = \frac{\alpha}{\epsilon} \left( \frac{E}{\epsilon} \right)^{-\gamma} $$
and the angle-integrated differential flux:
$$ \frac{dF}{dE} = 0.92 \; \pi \; \frac{dJ}{dE} $$
where we can set 0.92 to one if we want to approximate the earth as an
infinite plane.  The number of muons $N$ produced by a shower of energy $E$ also follows a power law:
\begin{equation}
N(E) = \beta \left( \frac{E}{\epsilon} \right)^{\eta}
\end{equation}
Suppose that these $N$ muons have lateral profile give by:
$$\Phi(r) = \Phi(0) \exp(-r / \lambda)$$
where $\lambda$ is the length scale of the muon lateral profile.  The condition:
$$\int_0^\infty \; \Phi(r) \, 2 \pi r \, dr = N$$
implies that:
\begin{equation}
\Phi(0) = \frac{N(E)}{2 \pi \lambda^2}
\end{equation}
and
\begin{equation}
\Phi(r) =  \frac{N(E)}{2 \pi \lambda^2} \; \exp(-r / \lambda)
\end{equation}
The radius at which the flux $\Phi$ drops to $\Phi_{\rm D}$ is:
$$r = -\lambda \ln \left( \frac{2 \pi \lambda^2 \Phi_{\rm D}}{N} \right)$$
which means the area $A_{\rm D}$ over which there is a detectable muon flux is give by:
$$A_{\rm D}(E) = \pi \lambda^2 \; \left( \ln \frac{2 \pi \lambda^2 \Phi_{\rm D}}{N(E)} \right)^2$$
which is only valid ($A_{\rm D} > 0$) when:
$$\Phi(0) = \frac{N(E)}{2\pi\lambda^2} > \Phi_D$$
which sets a minimum energy, as we shall see.  Using the power law, note that:
$$\frac{N(E)}{\Phi_{\rm D} \, 2\pi\lambda^2}
= \frac{\beta}{\Phi_{\rm D} \, 2\pi\lambda^2} \left( \frac{E}{\epsilon}\right)^{\eta} = \left( \frac{E}{\kappa \epsilon}\right)^{\eta}$$
where
\begin{equation}
\kappa = \left( \frac{\Phi_{\rm D} 2 \pi \lambda^2}{\beta}\right)^{1/\eta}
\end{equation}
and so:
\begin{equation}
A_{\rm D}(E) = \pi \lambda^2 \eta^2 \; \left( \ln \frac{E}{\kappa \epsilon} \right)^2
\end{equation}
which is only valid for $E > \kappa \epsilon$ (which you can verify is equivalent to requiring that $\Phi(0) > \Phi_{\rm D}$).

We are interested in the quanity:
\begin{equation}
\frac{dN}{dE dt} = F(E) A_{\rm D}(E) = (0.92) \pi \frac{dJ}{dE} A_{\rm D}(E)  
\end{equation}
Plugging in the results below we obtain our main result:
\begin{equation}
\frac{dN}{dE \, dt} = (0.92) \frac{\alpha}{\epsilon} \, (\pi \lambda \eta)^2 \, \left( \frac{E}{\epsilon}\right)^{\eta-\gamma}
\left( \ln \frac{E}{\kappa \epsilon}\right)^2
\end{equation}
which we need only integrate to calculate the rate at which a point on the surface of the earth has a local muon flux exceeding $\Phi_D$.  Defining:
\begin{equation}
f(\kappa) = \int_\kappa^\infty x^{\eta-\gamma} \left( \ln \frac{x}{\kappa}\right)^2 dx
\end{equation}
we can write the rate as:
\begin{equation}
\frac{dN}{dt} = (0.92) \alpha \, (\pi \lambda \eta)^2 f(\kappa)  
\end{equation}




\section{Introduction}



\begin{figure}[htbp]
    \centering
    \includegraphics[width=0.9\textwidth]{figs/eas.png}
    \caption{Extensive air showers, from Chapter 16 of ``Gaissner Cosmic Rays and Particle Physics''}
    \label{fig:eas}
\end{figure}

\begin{itemize}
 \item Using rates from CRDB.  
 \item Using Whitesons formula for the number of muons:
$$N_{\mu} = 0.028 (E / 1~{\rm GeV})^{0.93}$$ Note that for $E=10^{19}~\rm
   eV$, this formular predicts $5\times10^7$ in good agreement with
   LHS of Fig.~\ref{fig:eas}
 \item Muons have typical lateral extent of $\lambda$ greater than Moliere radius.  Assuming:
$$\rho(r) = \frac{N}{2\pi\lambda^2} \ \exp(-x/\lambda)$$
then $\lambda=220$ puts $\rho(1 {\rm km}) = 2 / \rm m^2$ as in figure. 
  
 \item Aim for $E \ge 10^{18}~\rm eV$.

 \item Expected rate is:
$$ R = \frac{1}{\rm km^2 \, day \, sr} \, \pi (\pi \lambda^2) = 0.44 \, / \, \rm day$$
 \item Total muons:
   $$N_\mu = 6.6 \times 10^6$$
 \item And density at center:
   $$\rho(0) = \frac{N_\mu}{2 \pi \lambda^2} = 21 /~\rm m^2$$
 \item $A\epsilon = 2\times 10^{-5}$, for 1000 phones:  $0.02~\rm m^2$
 \item Muons hits in phone per shower:  $0.42$ per shower.
 \item Expect a phone hit coincident with tagger more than once per week.
 \item Taggers of size $0.1 m^2$ should be well suited.
\end{itemize}

\end{document}
